\section{Experimental Prototype}

This section describes the implementation of the 3-user co-located VR training system using Meta Quest 3 headsets.

\subsection{Hardware Configuration}

The experimental testbed consists of:

\begin{itemize}
    \item \textbf{Headsets:} 3$\times$ Meta Quest 3 (Snapdragon XR2 Gen 2, 8GB RAM)
    \begin{itemize}
        \item Resolution: 2064$\times$2208 per eye
        \item Refresh rate: 90Hz (target), 72Hz (fallback)
        \item Tracking: Inside-out 6DoF with 4 wide-angle cameras
        \item Hand tracking: Optical tracking without controllers
    \end{itemize}
    
    \item \textbf{Network Infrastructure:}
    \begin{itemize}
        \item WiFi 6E access point (6GHz band, 160MHz channels)
        \item Dedicated router with QoS configuration
        \item Isolated network segment for minimal interference
    \end{itemize}
    
    \item \textbf{Physical Space:} 6m $\times$ 6m padded training area with Guardian boundaries
\end{itemize}

\subsection{Software Architecture}

The system is built on Unity 2022.3 LTS with the following components:

\begin{itemize}
    \item \textbf{Meta XR SDK v81.0.0:} Core VR functionality, hand tracking, passthrough
    \item \textbf{Photon Fusion 2.0.8:} State synchronization networking framework
    \item \textbf{Universal Render Pipeline (URP):} Performance-optimized rendering
    \item \textbf{Custom Building Blocks:}
    \begin{itemize}
        \item Colocation Manager: Spatial alignment and calibration
        \item SSA Manager: Shared spatial anchors for persistent world
        \item Avatar synchronization with NetworkBehaviour
        \item WIM interface implementation
    \end{itemize}
\end{itemize}

\subsection{Calibration Protocol}

Hand tracking-based alignment procedure:

\begin{enumerate}
    \item Each user places hands at three predefined spatial markers
    \item System captures 3D positions from each headset's coordinate frame
    \item Procrustes analysis computes optimal rigid transformation
    \item Validation measures Euclidean error at marker positions
    \item Target: $<10$mm accuracy (Reimer et al.~\cite{ReimerDennis2021CfSV})
\end{enumerate}

Real-time tracking quality monitoring triggers recalibration when accuracy degrades below threshold.

\subsection{Training Scenarios}

Three scenarios of progressive complexity:

\begin{description}
    \item[Scenario 1: Basic] Simple object manipulation and spatial coordination. Single shared task requiring turn-taking. Low network traffic, minimal rendering complexity.
    
    \item[Scenario 2: Medium] Multi-step procedures requiring verbal communication and spatial awareness. Multiple interactive objects, moderate avatar movement. WIM interface provides overhead tactical view.
    
    \item[Scenario 3: Complex] Time-critical emergency response simulation. Simultaneous parallel tasks, high coordination demands, complex spatial reasoning. Tests system limits under maximum load.
\end{description}

\subsection{World-in-Miniature (WIM) Interface}

The WIM spatial awareness tool provides:

\begin{itemize}
    \item Real-time 3D miniature representation of training environment
    \item Live avatar positions and orientations
    \item Interactive camera control (pan, zoom, rotate)
    \item Task-relevant object highlighting
    \item Wrist-mounted for continuous availability
\end{itemize}

Implementation based on Chen et al.'s~\cite{ChenLei2024Eoep} design principles, scaled to 1:10 environment ratio for optimal spatial comprehension.

\subsection{Data Collection}

Automated logging systems capture:

\begin{itemize}
    \item \textbf{Technical metrics:} Frame rate, frame time, network RTT, packet loss, device temperature (1Hz sampling)
    \item \textbf{Calibration data:} 3D tracking accuracy at marker positions (pre/mid/post session)
    \item \textbf{Task performance:} Completion time, error counts, success/failure, coordination events
    \item \textbf{Subjective measures:} Spatial awareness scores (post-task questionnaire)
\end{itemize}

All data timestamped and synchronized across headsets for correlation analysis.
